\documentclass[a4paper,12pt,titlepage]{article}

\usepackage[T1,T2A]{fontenc}     % форматы шрифтов
\usepackage[utf8x]{inputenc}     % кодировка символов, используемая в данном файле
\usepackage[russian]{babel}      % пакет русификации
\usepackage{tikz}                % для создания иллюстраций
\usepackage{pgfplots}            % для вывода графиков функций
\usepackage{geometry}		     % для настройки размера полей
\usepackage{indentfirst}         % для отступа в первом абзаце секции
\usepackage{listings}
\usepackage{amsmath}
\lstset{
basicstyle=\footnotesize,        % the size of the fonts that are used for the code
  breakatwhitespace=false,         % sets if automatic breaks should only happen at whitespace
  breaklines=false,                 % sets automatic line breaking
  captionpos=b,                    % sets the caption-position to bottom
  commentstyle=\color{green},    % comment style
  extendedchars=true,              % lets you use non-ASCII characters; for 8-bits encodings only, does not work with UTF-8
  keepspaces=true,                 % keeps spaces in text, useful for keeping indentation of code (possibly needs columns=flexible)
  keywordstyle=\color{blue},       % keyword style
  language=[95]Fortran,                 % the language of the code
  numbers=left,                    % where to put the line-numbers; possible values are (none, left, right)
  numbersep=5pt,                   % how far the line-numbers are from the code
  numberstyle=\tiny\color{gray}, % the style that is used for the line-numbers
  rulecolor=\color{black},         % if not set, the frame-color may be changed on line-breaks within not-black text (e.g. comments (green here))
  showspaces=false,                % show spaces everywhere adding particular underscores; it overrides 'showstringspaces'
  showstringspaces=false,          % underline spaces within strings only
  showtabs=false,                  % show tabs within strings adding particular underscores
  stepnumber=1,                    % the step between two line-numbers. If it's 1, each line will be numbered
  stringstyle=\color{blue},       % string literal style
  tabsize=4,                       % sets default tabsize to 2 spaces
  title=\lstname                   % show the filename of files
}
% выбираем размер листа А4, все поля ставим по 3см
\geometry{a4paper,left=30mm,top=30mm,bottom=30mm,right=30mm}

\setcounter{secnumdepth}{0}      % отключаем нумерацию секций

\usepgfplotslibrary{fillbetween} % для изображения областей на графиках

\begin{document}
% Титульный лист
\begin{titlepage}
    \begin{center}
	{\small \sc Московский государственный университет \\имени М.~В.~Ломоносова\\
	Факультет вычислительной математики и кибернетики\\}
	\vfill
	{\Large \sc Отчет по заданию №2}\\
	~\\
	{\large \bf <<Численное решение третьей краевой задачи для стационарного уравнения теплопроводности с непрерывными коэффицентами.\\
	    Метод прогонки.>>}\\ 
	~\\
    \end{center}
    \begin{flushright}
	\vfill {Выполнил:\\
	студент 305 группы\\
	Татосьян~В.~Г.\\
	~\\
	Преподаватель:\\
	Есикова~Н.~Б.}
    \end{flushright}
    \begin{center}
	\vfill
	{\small Москва\ 2020}
    \end{center}
\end{titlepage}

% Автоматически генерируем оглавление на отдельной странице
\tableofcontents
\newpage

\section{Постановка задачи}

\begin{itemize}
\item Найти решение обыкновенного дифференциального уравнения второго порядка с непрерывными коэффицентами, с двумя краевыми условиями третьего рода с точностью $\varepsilon = 0.01$: \newline
$\frac{d}{dx}[k(x)\frac{du}{dx}]-q(x)u=-f(x); 0 < x < 1; k(x) \geq c_1 > 0; q(x) \geq 0$ \newline
Краевые условия:\newline
$k(0)u'(0)=\beta_1u(0)+\mu_1\newline -k(1)u'(1)=\beta_2u(1)+\mu_2 $ \newline
где $\beta_1, \beta_2 \geq 0; \beta_1 + \beta_2 > 0; q(x), k(x), f(x)-$непрерывные функции
\item Для моего варианта: \newline
$k(x)=x^2+1; q(x)=x; f(x)=e^{-x};$\newline
$\beta_1 = 0; \beta_2 = 1; \mu_1=0; \mu_2 = 0; \varepsilon = 0.01$\newline
\item Реализация конечно-разностной схемы второго порядка апроксимации для решения данной задачи.
\item Значение шага $h$ для разностной схемы выбрать изходя из требований к точности решения. Решение разностной задачи искать методом прогонки.
\item Для оценки погрешности решения краевой задачи воспользоваться правилом Рунге.
\end{itemize}

\newpage

\section{Аналитическое решение задачи}

Решаем модельную задачу: $\bar k = k(0.5); \bar q = q(0.5); \bar f = f(0.5)$ \newline
Решение $u(x)$ обыкновенного диффернциального уравнения второго порядка выглядит следующим образом: \newline
$u(x)=C_1e^{\sqrt{\frac{q}{k}}x}+C_2e^{-\sqrt{\frac{q}{k}}x}+\frac{f}{q}$\newline
Вычисляя производную: \newline
$u(x)=C_1\frac{q}{k}e^{\sqrt{\frac{q}{k}}x}-C_2\frac{q}{k}e^{-\sqrt{\frac{q}{k}}x}$\newline
И подставляя краевые условия:\newline
$ku'(0)=0$\newline
$-ku'(1)=u(1)$\newline
Получим:\newline
$C_1=C_2=C=\frac{f}{q((\sqrt{kq}-1)e^{-\sqrt{\frac{q}{k}}}-(\sqrt{kq}+1)e^{\sqrt{\frac{q}{k}}})}$\newline


\section{Численное решение задачи}

Для численного решения задачи введем равномерную сетку $\omega$. \newline
$\omega = \{x_i = ih |\, i=[0,N], h = \frac{1}{N} \}$\newline
Заменим производную второго порядка конечно-разностным отношением вида: \newline
$\frac{d}{dx}(k(x)\frac{du}{dx}) = \frac{k(\frac{x_{i+1} - x{i}}{h}) - k(\frac{x_{i} - x{i-1}}{h})}{h}$\newline
Получим уравнение приближающее дифференциальное уравнение в узле:\newline
$k(x)\frac{y(x_{i+1})-2y(x_{i})+y(x_{i-1})}{h^2} - q(x)y(x_i)=-f(x)$ при $x \in (0, 1)$ \newline
Далее решаем систему линейных уравнений методом прогонки. \newline

\newpage
\section{Пример работы программы}

\begin{table}[h]
\centering
\begin{tabular}{|c|c|c|c|c|}
\hline
$i$ & $x_i$ & $u(x_i)$ & $y(x_i)$ & $delta$ \\
\hline
0&0&0.516205&0.516205&9.53146e-08\\
10&0.05&0.515857&0.515857&9.56527e-08\\
20&0.1&0.514811&0.514811&9.66675e-08\\
30&0.15&0.513067&0.513067&9.8361e-08\\
40&0.2&0.510623&0.510623&1.00736e-07\\
50&0.25&0.507476&0.507476&1.03798e-07\\
60&0.3&0.503624&0.503624&1.07551e-07\\
70&0.35&0.499062&0.499062&1.12003e-07\\
80&0.4&0.493786&0.493787&1.17163e-07\\
90&0.45&0.487791&0.487791&1.23039e-07\\
100&0.5&0.481071&0.481071&1.29642e-07\\
110&0.55&0.473618&0.473619&1.36986e-07\\
120&0.6&0.465426&0.465427&1.45082e-07\\
130&0.65&0.456487&0.456487&1.53947e-07\\
140&0.7&0.44679&0.446791&1.63596e-07\\
150&0.75&0.436328&0.436328&1.74048e-07\\
160&0.8&0.425088&0.425088&1.85321e-07\\
170&0.85&0.413061&0.413061&1.97436e-07\\
180&0.9&0.400233&0.400233&2.10416e-07\\
190&0.95&0.386593&0.386593&2.24283e-07\\
200&1&0.372126&0.372126&2.39063e-07\\
\hline
\end{tabular}
\caption{Значения аналитического решения $u(x)$ и численного $y(x)$ для модельной задачи\newlineпри $N=100$}
\label{table}
\end{table}

\newpage

\begin{table}[h]
\centering
\begin{tabular}{|c|c|c|c|c|}
\hline
$i$ & $x_i$ & $u(x_i)$ & $y(x_i)$ & $delta$ \\
\hline
0&0&0.516205&0.516205&9.56265e-10\\
100&0.05&0.515857&0.515857&9.59655e-10\\
200&0.1&0.514811&0.514811&9.69796e-10\\
300&0.15&0.513067&0.513067&9.86705e-10\\
400&0.2&0.510623&0.510623&1.0104e-09\\
500&0.25&0.507476&0.507476&1.04094e-09\\
600&0.3&0.503624&0.503624&1.07836e-09\\
700&0.35&0.499062&0.499062&1.12278e-09\\
800&0.4&0.493786&0.493786&1.17425e-09\\
900&0.45&0.487791&0.487791&1.23292e-09\\
1000&0.5&0.481071&0.481071&1.29889e-09\\
1100&0.55&0.473618&0.473618&1.37222e-09\\
1200&0.6&0.465426&0.465426&1.45305e-09\\
1300&0.65&0.456487&0.456487&1.54154e-09\\
1400&0.7&0.44679&0.44679&1.63783e-09\\
1500&0.75&0.436328&0.436328&1.74214e-09\\
1600&0.8&0.425088&0.425088&1.8547e-09\\
1700&0.85&0.413061&0.413061&1.97572e-09\\
1800&0.9&0.400233&0.400233&2.1054e-09\\
1900&0.95&0.386593&0.386593&2.244e-09\\
2000&1&0.372126&0.372126&2.39174e-09\\
1&1&1&0\\
\hline
\end{tabular}
\caption{Значения аналитического решения $u(x)$ и численного $y(x)$ для модельной задачи\newlineпри $N=1000$}
\label{table}
\end{table}

\newpage

\begin{table}[h]
\centering
\begin{tabular}{|c|c|c|}
\hline
$i$ & $x_i$ & $y(x_i)$ \\
\hline
0&0&0.600579\\
10&0.05&0.599363\\
20&0.1&0.595864\\
30&0.15&0.590319\\
40&0.2&0.582982\\
50&0.25&0.574114\\
60&0.3&0.563975\\
70&0.35&0.552818\\
80&0.4&0.540882\\
90&0.45&0.528388\\
100&0.5&0.515537\\
110&0.55&0.502505\\
120&0.6&0.489446\\
130&0.65&0.476491\\
140&0.7&0.463748\\
150&0.75&0.451308\\
160&0.8&0.439242\\
170&0.85&0.427603\\
180&0.9&0.416433\\
190&0.95&0.405762\\
200&1&0.395609\\
\hline
\end{tabular}
\caption{Значения численного $y(x)$ для общей задачи\newlineпри $N=100$}
\label{table}
\end{table}

\newpage

\begin{table}[h]
\centering
\begin{tabular}{|c|c|c|}
\hline
$i$ & $x_i$ & $y(x_i)$ \\
\hline
0&0&0.600574\\
100&0.05&0.599359\\
200&0.1&0.59586\\
300&0.15&0.590316\\
400&0.2&0.582979\\
500&0.25&0.574111\\
600&0.3&0.563972\\
700&0.35&0.552815\\
800&0.4&0.540879\\
900&0.45&0.528386\\
1000&0.5&0.515535\\
1100&0.55&0.502503\\
1200&0.6&0.489444\\
1300&0.65&0.476489\\
1400&0.7&0.463747\\
1500&0.75&0.451307\\
1600&0.8&0.43924\\
1700&0.85&0.427601\\
1800&0.9&0.416432\\
1900&0.95&0.405761\\
2000&1&0.395608\\
\hline
\end{tabular}
\caption{Значения численного $y(x)$ для общей задачи\newlineпри $N=1000$}
\label{table}
\end{table}


\newpage
\section{Исходный код}
\begin{lstlisting}[language=C++, caption={Исходный код программы}]
#include <iostream>
#include <cmath>


double estimationError (double * u, double * y, int N);
void realSolv(double * u, int N, double x0);
void sweepMethod(double * y, int N, double x0);
void generalProblem(int N);
void modelProblem(int N);


int main() {
    int N, model;

    std::cout << "Hello!" << std::endl;
    std::cout << "Please, choose which problem we will solve: model - 0, general - 1.\nEnter a number: ";
    std::cin  >> model;
    std::cout << "Please, enter N: ";
    std::cin  >> N;

    if (model == 0) {
        modelProblem(N);
    } else if (model == 1) {
        generalProblem(N);
    } else {
        std::cout << "When entering the first digit, choose between 0 and 1!\n 
            Please, restart program!" << std::endl;
    }
    
    return 0;
}


double estimationError(double * y2, double * y, int N) {
    double res;

    N = N / 2;

    for (int i = 0; i < N + 1; i++) {
        if (res < abs(y[2 * i] - y2[i]) / 3) {
            res = abs(y[2 * i] - y2[i]) / 3;
        }
    }

    return res;
}


void realSolv(double * u, int N, double x0) {
    double * x;

    double h = 1.0 / N;
    
    double k = x0 * x0 + 1;
    double q = x0;
    double f = exp(-x0); 

    double C = f / (q * ((sqrt(k * q) - 1) * exp(-sqrt(q / k)) 
        - (sqrt(k * q) + 1) * exp(sqrt(q / k))));
    x = new double [N + 1];

    for (int i = 0; i < N + 1; i++) {
        x[i] = i * h;
    }

    for (int i = 0; i < N + 1; i++) {
        u[i] = C * (exp(sqrt(q / k) * x[i]) + exp(-sqrt(q / k) * x[i])) + f / q;
    }

    delete [] x;
}


void sweepMethod(double * y, int N, double x0) {
    double * k;
    double * q;
    double * fi;
    double * x;
    
    double * alf;
    double * bet;
    double * a;
    double * b;
    double * c;

    double h = 1.0 / N;

    k = new double [N + 1];
    q = new double [N + 1];
    fi = new double [N + 1];
    x = new double [N + 1];

    alf = new double [N + 2];
    bet = new double [N + 2];
    a = new double [N + 2];
    b = new double [N + 2];
    c = new double [N + 2];

    for (int i = 0; i < N + 1; i++) {
        x[i] = i * h;

        if (x0 == 0.0) {
            k[i] = x[i] * x[i] + 1;
            q[i] = x[i];
            fi[i] = exp(-x[i]);
        } else {
            k[i] = x0 * x0 + 1;
            q[i] = x0;
            fi[i] = exp(-x0);
        }
    }

    for (int i = 1; i < N; i++) {
        a[i] = 1 / (2 * h * h) * (k[i - 1] + k[i]);
        b[i] = 1 / (2 * h * h) * (k[i] + k[i + 1]);
        c[i] = q[i] + 1 / (2 * h * h) * (k[i - 1] + 2 * k[i] + k[i + 1]);
    }

    alf[1] = (k[0] + k[1]) / (k[0] + k[1] + h * h * q[0]);
    bet[1] = h * h * fi[0] / (k[0] + k[1] + h * h * q[0]);

    for (int i = 1; i < N; i++) {
        alf[i + 1] = b[i] / (c[i] - a[i] * alf[i]);
        bet[i + 1] = (fi[i] + a[i] * bet[i]) / (c[i] - a[i] * alf[i]); 
    }

    double mu2 = 1 + 0.5 * h * q[N] + (k[N] + k[N - 1]) / (2 * h);
    double bet2 = (k[N] + k[N - 1]) / (2 * h * mu2);

    y[N] = (0.5 * h * fi[N] / mu2 + bet2 * bet[N]) / (1 - alf[N] * bet2);

    for (int i = N - 1; i >= 0; i--) {
        y[i] = alf[i + 1] * y[i + 1] + bet[i + 1];
    }

    delete [] k;
    delete [] q;
    delete [] fi;
    delete [] x;

    delete [] alf;
    delete [] bet;
    delete [] a;
    delete [] b;
    delete [] c;
}


void modelProblem(int N) {
    double * u;
    double * y;
    double * y2;

    int j = 0;

    double eps = 0.01, x0 = 0.5;

    while (N < 100000) {
        u = new double [N + 1];
        y = new double [N + 1];

        realSolv(u, N, x0);
        sweepMethod(y, N, x0);

        if (j == 1) {
            if (estimationError(y2, y, N) < eps) {
                for (int i = 0; i < N + 1; i += 50) {
                    std::cout << "x[" << i << "] = " << i * 1.0 / N << " , u[" << i << "] = " << u[i] << " , y[" <<
                        i << "] = " << y[i] << " , delta = " << abs(u[i] - y[i]) << std::endl;
                }

                delete [] u;
                delete [] y;
                delete [] y2;

                break;
            } else {
                delete [] y2;

                y2 = new double [N + 1];

                for (int i = 0; i < N + 1; i++) {
                    y2[i] = y[i];
                }

                N = 2 * N;
            }
        } else {
            y2 = new double [N + 1];

            for (int i = 0; i < N + 1; i++) {
                y2[i] = y[i];
            }

            N = 2 * N;
        }

        delete [] u;
        delete [] y;

        j = 1;
    }
}


void generalProblem(int N) {
    double * u;
    double * y;
    double * y2;

    int j = 0;

    double eps = 0.01, x0 = 0.0;

    while (N < 100000) {
        y = new double [N + 1];

        sweepMethod(y, N, x0);

        if (j == 1) {
            if (estimationError(y2, y, N) < eps) {
                for (int i = 0; i < N + 1; i += 50) {
                   std::cout << "x[" << i << "] = " << i * 1.0 / N << " , y[" <<
                        i << "] = " << y[i] << std::endl;
                }

                delete [] y;
                delete [] y2;

                break;
            } else {
                delete [] y2;

                y2 = new double [N + 1];

                for (int i = 0; i < N + 1; i++) {
                    y2[i] = y[i];
                }

                N = 2 * N;
            }
        } else {
            y2 = new double [N + 1];

            for (int i = 0; i < N + 1; i++) {
                y2[i] = y[i];
            }

            N = 2 * N;
        }

        delete [] y;

        j = 1;
    }
}
\end{lstlisting}
\end{document}